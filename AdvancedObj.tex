\chapter{面向对象}
结构化编程,重点在编写函数,处理数据,数据和函数分开定义,松散关联。面向对象编程,重点在创建对象。对象既包含数据又包含处理数据函数,二者定义放入同一个数据结构中,密不可分。这个数据结构称为\emph{类}。程序运行时,依据类中的定义,创建对象。

面向对象程序设计技术源于对隐藏信息的需要。结构化编程与面向对象编程,如海大和吉大图书馆,分别采用开放数据与控制访问制度。书就像数据,控制访问情况下,只能通过馆员借还书,绝对不允许自己动手。面向对象程序中,对象的数据称为\emph{属性},对象的函数称为\emph{方法}。这样的程序中,数据仿佛被隐藏和保护起来,访问受到严格控制,只能通过特定函数读取、修改数据。这种隐藏和保护数据术语称为\emph{信息封装}。

\section{一切皆对象}
Python支持面向对象编程,将数据和操作数据的函数抽象为一整体,用术语\emph{对象}来刻画这个整体。每个对象有唯一的标识、属于某种数据类型,有值。Python中所有事物都是对象。列表是对象,42是对象,模块是对象,函数是对象。
\begin{quotation}
  One of my goals for Python was to make it so that all objects were ``first class."  By this, I meant that I wanted all objects that could be named in the language(e.g., integers, strings, functions, classes, modules, methods, and so on) to have equal status. That is, they can be assigned to variables, placed in lists, stored in dictionaryes, passed as argumetns, and so forth.''
  --Guido van Rossum(Blog, The History of Python, February 27, 2009)
\end{quotation}
\subsection{对象标识}
\pyth{id(object)}显示一个对象的标识。
\begin{python}
def foo(): pass

type(foo), id(foo)
# (<class 'function'>, 38110760)   #  function is an object
type(foo.__code__)
id(foo.__code__)
# (<class 'code'>, 38111680)  #  code is an object
\end{python}
\subsection{对象的数据类型}
对象的数据类型定义了对象数据的取值范围,同时定义了对象可以完成的操作。数据类型本身也是对象,它的数据类型就是自身。
\begin{python}
type(42)
#  <class 'int'>
type(type(42))
#  <class 'type'>
type(type(type(42)))
#  <class 'type'>
def f(x):
    return x+1
type(f)
import math
type(math)
\end{python}

\section{类}
类是用户自定义对象的数据类型,是创建对象的模板。定义类语法如下。
\begin{framed}
\begin{verbatim}
  class 类名(父类1, 父类2...):
      """类说明"""
      def __init__(self, arg1, arg2...):
          """方法说明"""
          构造器方法体

      def method1(self, arg1, arg2...):
          """方法说明"""
          方法1语句

      ……
\end{verbatim}
\end{framed}
下面例子用类的语法,创建一个学生类。
\begin{python}
class Student(Human): 
    def __init__(self, name='Adam', sex='Male', age=18):
        self.name = name
        self.sex = sex
        self.age = age

    def ChangeName(self, new_name):
        self.name = name
         
s = Student('Mary', 'Female, 19)
type(s)
type(type(s))
\end{python}
上面例子中,按Python语法,创造了自定义数据类型学生类。术语“类”和“数据类型”是同一个概念的两个名字。为防混淆,一般称内置数据类型为数据类型,称自定义数据类型为类。类合数据和操作为一,提供创造新数据类型的机制。创建一个类,就是创建了某种对象的新数据类型。类有属性,描述状态,代表类的数据,表示类的静态方面;类有方法,修改状态,代表了类的行为,表示类的动态方面。
\subsection{实例}
实例(instance)是对象的同义语。 “实例是某数据类型的对象”。例如,42是int数据类型的实例,即42是一个整型对象。上例中,s是学生类的实例,是学生对象,用来表示一个具体的学生Mary,其数据类型是学生类。

回想下,我们已在前面学习中,接触到类和对象了。比如,列表是类,具体的列表是列表实例,而列表方法,正是该类的方法。
\subsection{构造器}
\pyth{__init__()}称为\emph{构造器},该方法用于创造一个实例化的类——对象。该方法是创建对象后,第一个调用的方法。
\pyth{__del__()}称为\emph{析构器},该方法用于摧毁对象。由于Python有支持垃圾收集技术(garbage collection),析构器几乎很少使用。
\subsection{增加属性和方法}
属性描述类的性质。类的不同实例初始化为对象后,有各自的属性值,对象属性值描述每个对象的状态。属性与变量一样,赋值即定义。属性赋值语句:\pyth{object.attribute = value}。习惯上,在构造函数内,声明对象属性并赋值。

方法描述类的行为,指明类的对象能做哪些事情。对象方法做的主要事情,就是改变对象的属性值,即改变对象的状态,完成从一个状态到另一个状态的转移。方法和函数的区别在于,方法必须依附于某个对象。因此,方法的所有参数中,第一个参数必须是调用该方法的对象,即该方法依附的对象。习惯上,这个参数命名为self。

一般,属性指对象的属性,方法指对象的方法。后面还要学习类属性、类方法,它们与对象的属性和方法不同。
\subsection{读取属性和调用方法}
读取属性语法为\pyth{object.attribute},调用方法语法为\pyth{object.method(arg1,arg2...)}。有同学可能会注意到,调用方法时,参数列表中第一个参数并没有提供self对象。这如何解释?可以这样理解\pyth{mehtod(object, arg1, arg2)}。这也再次说明了方法与函数的差别。
\inputpython{\ooeg/init.py}{1}{30}
\subsection{练习:创造类}  
二维平面中,视点为一个实体,其属性为直角坐标系下横纵坐标,如(0, 0)、(x, y)。与点有关的操作有,计算该点到原点距离,计算该点到某点的两点间距离,计算两点的中点坐标,或该点是否在某个矩形或圆形范围内等等。

尝试自定义类Point,代表二维平面上的点Point,该类有两个属性,分别代表横纵坐标。方法有:构造器;移动,修改点的坐标;重置,将点坐标设置为原点;计算距离,如果已知另一个点,则可计算其与本点的欧式距离。
\inputpython{\ooeg/Point.py}{1}{45}  

\section{对象特征}
面向对象语言提供了对一些强大特征的支持。结构化语言不支持这些特征。
\subsection{信息封装}
信息封装是一种信息访问保护机制,属性和方法的封装类型一般分三个层次。
\begin{table}
  \centering
  \begin{tabular}{ccc}
    \toprule
    属性和方法名 & 类型 & 类型含义
    \midrule
    name        & 公开  & 常规命名的属性和方法供所有类自由访问、调用,没有限制
    \_name      & 保护  & 前缀单下划线的属性和方法,供该类及其子类使用访问,其他类不应该访问
    \_\_name    & 私有  & 前缀双下划线的属性和方法,仅供该类自己使用,其他类绝不应该访问
    \bottomrule
  \end{tabular}
  \caption{属性封装类型}
\end{table}
从技术角度看,Python在语言层面没有提供严格访问保护机制,所有属性和方法本质上都是公开的。 程序中,如果该属性供内部使用,不希望其他类访问,应在文档中注明,这是一个内部方法。或者,按大多数Python程序员的习惯做法,修饰名称(name mangling)。这个术语指这样一种技术,通过给属性或方法名,加一个下划线前缀,来标记这是一个应该隐藏的属性。更强烈的标记是前缀双下划线。事实上,访问双下划线属性也是可以的,只不过写法上费点事。如此费事,岂不正是程序员强烈提醒用户不要使用此属性或方法。另外,双下划线,是许多特名(special names)的标志,比如构造器。因此,程序员给属性、方法起名,为了标识信息隐藏,一个下划线足够,尽量不要使用双下划线。
\inputpython{\ooeg/dunder.py}{1}{18}
\subsection{继承}
继承是重用代码的一种方法。继承允许在两个乃至多个类之间建立种属、上位类-下位类、继承(is a)关系。通用逻辑代码放到父类中,细节逻辑代码放到子类中。Python中每个类都使用了继承关系,所有类都继承自一个特殊类object。这个类几乎没有实用功能。但它提供了统一的视角,看待所有Python类。大家都是object类的子类。

\fbox{\pyth{inheri}}

Supplier可以做Contact能做的所有事情,及他自己的事情。这就是继承的优势,子类继承父亲的属性和行为,同时增加自己特有的属性和行为。比如,s的初始化,调用的是Contact的初始方法。Supplier继承了Contact的属性和方法。
\begin{framed}
  \begin{verbatim}
- 继承原则
  + Distinguish Interface Inheritance from Implementation Inheritance
    * /Inheritance of interface/ creates a subtype, implying an "is-a"
      relationship
    * /Inheritance of implementation/ avoids code duplication by reuse
  + Make Interfaces explicit with ABCs
  + Use Mixins for code Reuse
    *  Conceptually, a mixin does not define a new type; it merely
       bundles methods for reuse.  A mixin should never be
       instantiated, and concrete classes should not inherit only from
       a mixin.  Each mixin should provide a single specific behavior,
       implementing few and very closely related methods.
  + Make Mixins Explicit by Naming
  + An ABC may also be a mixin; the reverse is not true
  + Dont't subclass from more than one concrete class
  + Provide aggregate classes to users
    * If some combination of ABCs or mixins is particularly useful to
      client code, provide a class that brings them together in a
      sensible way
  + Favor object composition over class inheritance, don't overuse
    inheritance.  Because subclassing is a form of tight coupling, and
    tall inheritance trees tend to be brittle. 
\end{verbatim}
\end{framed}

\fbox{需要一个解释继承的例子}

\subsection{多态}


\section(高级技术?)
\subsection{Property}
有的面向对象语言,把读取和修改属性的方法称为getter和setter方法,统称为访问器。原则上,要求每个属性配备一对访问器,通过访问器方法,对属性和方法进行控制访问,从而达到信息封装目的。
\input{ooeg/gsetter.py}{1}{46}

Python同时,提供了另一种信息封装技术,替代访问器机制。该技术通过关键字property,将访问器包装成属性,使得访问器方法看上去和属性一样,再利用包装后的属性控制读取和修改。
\begin{python}
class Rectangle:
    def __init__(self):
        self.width = 0
        self.height = 0
    def setSize(self, size=(0,0)):
        self.width, self.height = size
    def getSize(self):
        return self.width, self.height
# old syntax
    size = property(getSize, setSize)

# use setter and getter
r = Rectangle()
r.width = 10
r.height = 5
r.getSize()
r.setSize((150, 100))
r.width

# use property
r = Rectangle()
r.width = 10
r.height = 5
r.size
r.size = 150, 100
r.width
\end{python}
例子中size依然使用的是getSize, setSize读写属性。但从用户角度看,size就像一个属性一样,访问细节被隐藏起来。

\begin{python}
# class Color:
#     def __init__(self, rgb, name):
#         self._rgb = rgb
#         self._name = name

#     def set_name(self,name):
#         self._name = name

#     def get_name(self):
#         return self._name

# c = Color("#ff0000", "bright red")
# print(c.get_name())
# c.set_name("red")
# c.get_name()

class Color:
    def __init__(self, rgb, name):
        self.rgb = rgb
        self._name = name

    def _set_name(self, name):
        if not name:
            raise Exception("Invalid Name")
        self._name = name

    def _get_name(self):
        return self._name

    name = property(_get_name, _set_name)

c = Color("#ff0000", "bright red")
print(c.name)    # getter method is like attribute
c.name = "red"   # setter method is like attribute
print(c.name)
c.name = ""
\end{python}

\subsection{细究property}

\pyth{property()}是一个内置函数,返回一个property对象。其函数签名如下。
\begin{itemize}
\item \pyth{property(fget=None, fset=None, fdel=None, doc=None)}
\item fdel用于记录值被删除时情况
\item docstring是说明,解释该property
\end{itemize}
通过一个例子看下这几个参数都是什么意思。
\begin{python}
class Silly:
    def _get_silly(self):
        print("Your are getting silly")
        return self._silly

    def _set_silly(self, value):
        print("You are making silly {}".format(value))
        self._silly = value

    def _del_silly(self):
        print("Whoah, you killed silly!")
        del self._silly

    silly = property(_get_silly, _set_silly, _del_silly, "This is a silly property")

s = Silly()
s.silly = "funny"
s.silly

help(s)
del s.silly
\end{python}
试着将\pyth{property()}想象成一位代表,代表某个属性。有了property,用户想读取或修改属性值,就不必直接调用访问器方法了,而是去联系它的代表,由代表来解决读取和修改。用户只需从代表那里,得到自己想要的结果。

实际上,访问器方法依然需由程序员完成,property对象依然通过这两个方法读写属性。property的作用是将这两个方法对用户隐藏起来,户不用关心读写方法了。同时,对于不需要用户读取或修改的属性,就不再提供访问器方法,用户也无法通过property访问了。这既降低了代码冗余,也适当隐藏了部分信息。

技术上,property是类的实例,是property对象。该类用“神奇方法(magic methods)”,正式名称为“特别方法(special names)”的技术完成工作。这些方法是\pyth{__get__(),__set__(), __delete__()}。

以上三个方法定义了描述器协议(descriptor protocol)。一个对象只要实现了这三个方法中任一个,就可称为描述器(descriptor)。描述器特别处在于,它可以控制对属性的读写访问。而完成读写属性的方法,正是那三个特别方法。更多内容,请阅读: \url{https://docs.python.org/3/howto/descriptor.html}。

\subsection{装饰器:另一种创建property方法}
新的property语法用装饰器(decorator)技术,返回property对象。装饰器通过语法糖衣(syntax sugar, constuctor like []),将@property放在方法定义上一行,表明这个方法某属性的描述器方法。这样做,免去给私有属性加下划线的麻烦,又将多个方法整合为一个property。如此装饰发生,有一个要求,所有方法必须同名。此外,property的docstring由getter方法中的注释指定。
\begin{python}
class Silly:
    @property
    def silly(self):  # getter
        "This is a silly property."  # docstring
        print("You are getting silly")
        return self._silly

    @silly.setter
    def silly(self, value): #  setter
        print("You are making silly {}".format(value))
        self._silly = value

    @silly.deleter
    def silly(self):  # deleter
        print("Whoah, you killed silly!")
        del self._silly
# Note all the methods share a common name.
s = Silly()
s.silly = "funny"
s.silly

help(s)
del s.silly
\end{python}
该类的property采用现代写法,与前面行为一致,可读性更好了,代码看上去也更优雅。
\subsection{method, attribute, property辨析}
如何选择使用属性、方法和property?property模糊了属性和方法的界限。从技术角度看,Python中数据属性、property、方法,都是类的属性。尽管方法是可调用的,它依然是属性。因此可以这样理解,数据是数据属性,方法是可调用属性,而property是定制属性。这两点,可以帮助程序员决定,设计类时,应该选用哪个。

方法代表动作,由对象发出的动作,或施动于对象的动作。方法要做事情,方法名字一般是动词。

如果类的属性不是动作,那就要决定,是选用标准数据属性,还是property。一般都使数据属性,数据属性是名词。若需要对数据属性进行控制访问,用property。数据属性和property区别就是,当存取property时,会调用特定的方法访问。

\fbox{manager例子有用?}
\inputpython{examples/ooeg/manager.py}{1}{120}

\subsection{类属性和类方法}
类方法(class methods)和静态方法(static methods)由装饰器@staticmethod和@classmethod,包装对象方法生成。类方法用cls指代该类,就像self指代某对象一样。
\fbox{区别,何时用类方法,何时用静态方法?}
% - 阅读
%   + https://julien.danjou.info/blog/2013/guide-python-static-class-abstract-methods 
%   + https://stackoverflow.com/questions/136097/what-is-the-difference-between-staticmethod-and-classmethod-in-python
%   + https://stackoverflow.com/questions/12179271/meaning-of-classmethod-and-staticmethod-for-beginner
%   + https://www.geeksforgeeks.org/class-method-vs-static-method-python/
% - 例子
\begin{python}
class MyClass:
    @staticmethod
    def smeth():
        print("This is a static method")

    @classmethod
    def cmeth(cls):
        print("This is a class method of", cls)

MyClass.smeth()
# This is a static method
Myclass.cmeth()
# This is a class method of <class '__main__.MyClass'>
\end{python}
\subsection{静态方法}
Python中的静态方法主要是方便将外部函数集成到类体中, 重点在不需要类实例化的情况下调用方法。
\begin{python}
# example
# IND = 'ON'

# class Kls(object):
#     """"""
#     def __init__(self, data):
#         """"""
#         self.data = data

#     @staticmethod
#     def checkind():
#         """"""
#         return (IND == 'ON')

#     def do_reset(self):
#         """"""
#         if checkind():
#             print('Reset done for: {}'.format(self.data))

#     def set_db(self):
#         """"""
#         if checkind():
#             self.db = 'New db connection'
#             print("DB connection made for: {}".format(self.data))

# ik1 = Kls(12)
# ik1.do_reset()
# ik1.set_db()
# print(ik1.checkind())
# print(Kls.checkind())
\end{python}

\subsection{对象表示法}
用UML做面向对象分析与设计。自学
\href{https://en.wikipedia.org/wiki/Unified_Modeling_Language}{United Modeling Language}
\href{http://www.uml.org/what-is-uml.htm}{统一建模语言}
\href{https://www.w3cschool.cn/uml_tutorial}{UML tutorial}
\href{file:./misc/UML_Tutorial.pdf}{UML参考手册}
\subsection{练习: 简易记事本}
UML设计图a simple command-line notebook application
\subsubsection{Analysis}
notes are short memos stored in a notebook.  Each note should record
the day it was written and can have tags added for easy querying.  It
should be possible to modify notes.  We also need to be able to search
for notes.  All these things should be done from the command line.
\subsubsection{Design}

Note object.  A notebook container object.  Tags and dates also seemto
be objects, but we can use dates from Python's standard library.

Note objects have attributes for memo, tags, creation\_date, and id.  A
match method for searching Notebook has the list of notes as an
attribute.  A search method that returns a list of filtered notes.
interact with these objects? Implement the command-line menu interface

\subsection{继承}
继承是重用代码的一种方法。继承允许在两个乃至多个类之间建立种属、上位类-下位类、继承(is a)关系。通用逻辑代码放到父类中,细节逻辑代码放到子类中。Python中每个类都使用了继承关系,所有类都继承自一个特殊类object。这个类几乎没有实用功能。但它提供了统一的视角,看待所有Python类。大家都是object类的子类。

\fbox{\pyth{inheri}}

Supplier可以做Contact能做的所有事情,及他自己的事情。这就是继承的优势,子类继承父亲的属性和行为,同时增加自己特有的属性和行为。比如,s的初始化,调用的是Contact的初始方法。Supplier继承了Contact的属性和方法。
\begin{framed}
  \begin{verbatim}
- 继承原则
  + Distinguish Interface Inheritance from Implementation Inheritance
    * /Inheritance of interface/ creates a subtype, implying an "is-a"
      relationship
    * /Inheritance of implementation/ avoids code duplication by reuse
  + Make Interfaces explicit with ABCs
  + Use Mixins for code Reuse
    *  Conceptually, a mixin does not define a new type; it merely
       bundles methods for reuse.  A mixin should never be
       instantiated, and concrete classes should not inherit only from
       a mixin.  Each mixin should provide a single specific behavior,
       implementing few and very closely related methods.
  + Make Mixins Explicit by Naming
  + An ABC may also be a mixin; the reverse is not true
  + Dont't subclass from more than one concrete class
  + Provide aggregate classes to users
    * If some combination of ABCs or mixins is particularly useful to
      client code, provide a class that brings them together in a
      sensible way
  + Favor object composition over class inheritance, don't overuse
    inheritance.  Because subclassing is a form of tight coupling, and
    tall inheritance trees tend to be brittle. 
\end{verbatim}
\end{framed}

\fbox{需要一个解释继承的例子}
    
% **** 通过继承扩展内置数据类型

% - python所有类都继承自一个特殊类object
% - 如果自定义类继承自其他数据类型,会怎样?比如上例,如果想在通讯列表中
%   搜索姓名,要怎么办?
% - 可以在Contact类中增加Search方法,但Search似乎应该是附于list的方法,而
%   不是附于Contact
% - 用继承可以解决这个问题
% - pythonic方式是使用duck typing,定义Contact为iterable,本例为讲继承而
%   使用继承机制

% #+begin_src python :results output
% class ContactList(list):
%     """
%     Contactlist是list子类,因此继承list的方法,并增加了list没有,而自己
%     独有的方法search()
%     """
%     def search(self, name):
%         """Return all contacts that contain the search value in their name."""
%         matching_contacts = []
%         #  the syntax [] is called syntax sugar that calls the list() constructor.
%         # [] == list()
%         # isinstance([], object)
%         for contact in self:
%             if name in contact.name:
%                 matching_contacts.append(contact)
%         return matching_contacts


% class Contact:
%     all_contacts = ContactList()  #  类变量,是一个扩展的list

%     def __init__(self, name, email):
%         self.name = name
%         self.email = email
%         Contact.all_contacts.append(self)


% class Supplier(Contact):
%     def order(self, order):
%         print("If this were a real system we would send '{}' order to '{}'".format(order, self.name))


% c1 = Contact("John A", "johna@ex.net")
% c2 = Contact("John B", "johnb@ex.net")
% c3 = Contact("Jenna C", "jennac@ex.net")

% [c.name for c in Contact.all_contacts.search("John")]
% #+end_src

% - 可以扩展python的各种内置数据类型,为自己项目所用,比如:字典
% - 标准的python用法,不直接继承built-in type
% - collections模块提供了Userdict, UserList, UserString。想继承内置类型性质,继承这几个
% - fluent python中讲了继承内置类型可能引发的问题

% #+begin_src python :results output
% # 仅示例,不推荐这种用法
% class LongNameDict(dict):
%     """继承字典,既包括字典属性和方法,又增加独有方法longest_key()"""
%     def longest_key(self):
%         longest = None
%         for key in self:
%             if (not longest) or (len(key) > len(longest)):
%                 longest = key
%         return longest

% longkeys = LongNameDict()
% longkeys['hello'] = 1
% longkeys['longest yet'] = 5
% longkeys['hello2'] = 'world'

% longkeys.longest_key()
% #+end_src
\subsubsection{覆盖和父类初始化函数}
覆盖是继承机制中很重要的一项技术,对于构造函数尤其重要。下面例子里,子类通过覆盖,继承了父类方法名字,但改变了父类的行为。
\inputpython{examples/ooeg/override1.py}{1}{14}

再看一例,覆盖父类的构造函数。
\inputpython{examples/ooeg/override1.py}{1}{29}

继承可以让子类在父类基础上增加行为,可否继承并改变父类的行为呢?比如Contact,如果想给特殊的联系人,如朋友,增加电话号码怎么办?创建一个继承Contact的子类Friend,使其包含phone属性,是一个办法。
\begin{python}
class Friend(Contact):
    def __init__(self, name, email, phone):
        self.name = name
        self.email = email
        self.phone = phone
\end{python}
子类Friend的\pyth{__init__()}方法,覆盖了父类Contact的\pyth{__init__()}方法。覆盖意指,用一个名字相同、内容不同的新方法调整或替换父类中的方法。Friend继承了Contact的属性和方法,但使用覆盖技术调整了\pyth{__init__()}方法。

这个类,有一个问题,Contact和Friend重复了某些代码,比如email要改为Email或e\_mail,那么维护程序,需要修改两处代码。另一个更严重的问题是,Friend的\pyth{__init__()}方法没有把Friend对象加到联系人列表all\_contacts,逻辑上说不通。正确的做法,或说正确的子类初始化函数应该先调用父类的初始化函数,再做子类的初始化工作,比如增加属性。\pyth{super()}指该类的直接父类。
\begin{python}
class Friend(Contact):
    def __init__(self, name, email, phone):
        super().__init__(name, email)  # call super's constructor
        self.phone = phone
\end{python}
\subsubsection{多重继承}
单重继承指一个子类只有一个直接父类,多重继承指一个子类有多个直接父类。Java是典型的单重继承,C++、Python是典型的多重继承。
\paragraph{混入类}
最简单和最常用的多重继承形式,是所谓混入 (mixin) 类。混入类,是一个承担父类功能的类,它一般不独立使用,而是指望它被其他类继承,由此,该父类的功能被扩展到其他类。

假设,我们想让Contact类有发送电子邮件的功能。此外,还应考虑到,不只Contact,其他类也可能需要发送电子邮件的功能,那么就创建一个能发送电子邮件的混入类。其他类,谁想包含发电邮功能,继承此类即可。
\inputpython{examples/ooeg/mixin.py}{1}{12}
\paragraph{菱形难题}
当使用多重、多层继承时,多重继承会越来越麻烦。会遇到所谓菱形难题(diamond problem)。

例如,要给Friend增加地址属性。地址是一个字符串,包含街、市、省、国等信息。我们有几个选择。
\begin{itemize}
\item 可以把这些信息视为多个字符串参数,送入Friend初始化函数
\item 也可以把它们组成元组和字典,作为一个参数发给初始化函数
\item 更可以单独建立一个Address类,这些字符串都是类的属性,把Address的实例作为一个参数发给Friend初始化函数
\end{itemize}
第三种选择是将地址抽象为一个类,其优势,一是可以增加方法,比如给出导航路线等。另一个优势,是其他需要地址的类,比如建筑物类、企业类、组织类可以重用地址类。通过组合两个类方式得到新类,技术上称为组合 (composition)。组合标志两个类之间建立了属于(has a)关系。

为了研究继承,我们使用继承来实现地址,创建一个AddressHolder类。
A Friend can have an Addres. Also, a Friend is an AddressHolder.

菱形难题怎么发生的呢?两个父类的都有\pyth{__init__()}方法。当这两个类都继承自object就出问题了。基类仅应被调用一次,那么,调用顺序是Friend->Contact->AddressHolder->Object,还是Friend->Contact->AddressHolder->Object?
\paragraph{理解MRO}
技术上,调用顺序称为方法消解顺序(MRO, method resolution order)。根据MRO,可以找到解决菱形难题的方法。
例1:
\inputpython{examples/ooeg/diamond1.py}{1}{36}
例2:
\inputpython{examples/ooeg/diamond2.py}{1}{37}
通过输出结果看到,基类被调用了两次。这能引致错误,比如同一个银行账户,取或存两次。
\paragraph{菱形难题解决问题方法之一是使用super}
\inputpython{examples/ooeg/diamond3.py}{1}{37}
\begin{itemize}
\item Subclass的\pyth{super().call_me()}调用\pyth{LeftSubclass.call_me()},进而调用其\pyth{super().call_me()}
\item 但在这里super()没有指向LeftSubclass的基类BaseClass,而是指向RightSubclass, 尽管它不是LeftSubclass的父类。RightSubclass再调用BaseClass的\pyth{call_me()}
\item 因此,多重继承super()找的并不是“父辈”方法,而是“后继(next)”方法
\item 这个顺序是内置的mro机制决定的
\item 结论:减少,乃至避免使用多重继承,除非有必要。调用父类方法时,不用直接调用语句,使用super()
\item 实践上,duck typing的使用极大的减少了多重继承,乃至继承的使用必要
\end{itemize}
\subsection{参数分组}
参数分组,会让多重继承问题进一步复杂。再看到Friend多重继承例子。其\pyth{__init__()}方法里,参数被分组,分别传给了它的两个父类。
\begin{python}
class Friend(Contact, AddressHolder):
    def __init__(self,
                name,
                email,
                phone,
                street,
                city,
                state,
                code):
        Contact.__init__(self, name, email)  # 1 group arguments
        AddressHolder.__init__(self, street, city, state,code)  # 2 groups of arguments
        self.phone = phone
\end{python}
按上节原则,调用父类方法时,不直接调用,而使用super(),那如何将参数分为不同组,传给super()呢?比如,super()先将name、 email传给\pyth{Contact.__init__()},然后\pyth{Contact.__init__()}调用super,这就需要把地址信息传给“后继”方法,即\pyth{AddressHolder.__init__()}。当调用同名、不同参数的父类方法时,就会出现这个问题。最常出现这种问题处就是\pyth{__init__()}方法。唯一解决办法,就是预先计划。设计基类参数列表时,基类用,而不一定每个子类都用的参数,定义为关键字参数。同时,定义该方法可以接受可变参数,并将这些参数传给它的super()方法。
  
看下面较正规的多重继承代码。
\inputpython{examples/ooeg/multiinheri.py}{1}{22}
希望同学们在阅读和编程实践中逐步提高,掌握多重继承的使用,理解其代码和涵义。
\subsection{Polymorphism}
多态(polymorphism)概念指,当不知道调用方法的子类是哪个时,系统会自动根据子类所属类别,执行相应的方法,产生不同的行为。

假设用程序播放音频。一个播放器对象(media player object)创建AudioFile对象,执行play方法。play()方法负责解码音频文件,并播放,其行为可简单描述为audio-file.play()。但解码音频文件,需要根据不同的文件使用不同的算法,wav,mp3,wma,ogg,显然不能用同一种算法解码。难道要为每种文件定义一个play()方法嘛,那得定义多少个方法,定义多少个方法够用呢? 随着技术发展,出现新的音频文件格式怎么办?使用继承和多态可以简化设计。每种格式用AudioFile的一个子类表示,如WavFile, MP3File。每个子类都有自己的play()方法。但该方法根据不同子类实现不同的解码算法和播放功能。而播放器对象永远不必知道AudioFile指的是哪个子类,播放器只需调用play()方法,让对象的多态机制处理具体的解码播放即可。

示例如下,该例涵盖继承、覆盖、多态技术
\inputpython{examples/ooeg/poly.py}{1}{112}

\subsection{Duck Typing}
\href{http://en.wikipedia.org/wiki/Duck_typing}{Duck Typing}是python处理多态的方式,是一种不严格的多态。DT允许非子类的对象也可以调用多态方法。
\href{http://www.voidspace.org.uk/python/articles/duck_typing.shtlm}{duck typing的解释}

% #+begin_example
% Don't check whether it is-a duck: check whether it quacks-like-a duck, walks-like-a 
% duck, etc, etc, depending on exactly what subset of duck-like behavior you need to play
% you language-games with.(comp.lang.python, Jul. 26, 2000)

%      --Alex Martelli

% If it looks like a duck and quacks like a duck, it must be a duck.

% Duck typing in Python allows us to use /any/ object that provides the
% required behavior without forcing it to be a subclass。
% #+end_example

用“鲸鱼”、“鳄鱼”解释duck typing,fishable-like whale。由于海棠花栽培历史悠久,深受人们喜爱,后来人们发现了与它花形相似的植物,往往就以海棠来命名,如秋海棠、四季海棠,丽格海棠等等。明代《群芳谱》所记载的“海棠四品”: 西府海棠,蔷薇科苹果属;垂丝海棠,
蔷薇科苹果属;贴梗海棠,蔷薇科木瓜属;木瓜海棠,蔷薇科木瓜属。

在oop中广泛使用继承是因为有多态存在。在python中,由于duck typing存在,降低了使用多态的必要程度,同时也降低了多重继承的必要程度。

\begin{python}
import collections

Card = collections.namedtuple('Card', ['rank', 'suit'])

class FrenchDeck:
    ranks = [str(n) for n in range(2, 11)] + list('JQKA')
    suits = 'spades diamonds clubs hearts'.split()

    def __init__(self):
        self._cards = [Card(rank, suit) for suit in self.suits
                                        for rank in self.ranks]

    def __len__(self):
        return len(self._cards)

    def __getitem__(self, position):
        return self._cards[position]

beer_card = Card('7', 'diamonds')
beer_card

deck = FrenchDeck()
len(deck)

deck[0]
deck[-1]

from random import choice 
choice(deck)

deck[:3]
deck[12::13]

# for card in deck:  #  doctest: +ELLIPSIS
#     print(card)

# for card in reversed(deck):
#     print(card)
\end{python}

% **** 虚基类

% - 与dock typing常伴生的一项技术是虚基类 (ABCs, Abstract Base Classes)。
%   不展开,留自学
% - 大部分虚基类都放在python标准库的collections.abc模块中
% - https://docs.python.org/3/library/collections.abc.html?highlight=collections%20abc#module-collections.abc
% - 例:

% #+begin_src python :results output
% from collections import Container
% Container.__abstractmethods__
% #  frozenset({'__contains__'})
% help(Container.__contains__) 

% # 使用虚基类
% class OddContainer:
%     def __contains__(self, x):
%         if (not isinstance(x, int)) or (not x%2):
%             return False
%         return True

% oc = OddContainer()
% #  类型判断
% #  某个自定义类,duck typing了某个虚基类的protocol,就可以断定它是该虚基类的实例或子类。
% #  勿须继承自该虚基类
% isinstance(oc, Container)
% issubclass(OddContainer, Container)
% #+end_src
不通过继承(或多重继承),dt创建了一种is a关系。



% *** 案例2: 房地产管理应用程序

% - 创建一个房地产管理应用,让中介管理出售和出租的资产
%   + 有两种资产:公寓和独栋。中介录入资产信息,列出所有可用资产,标记资
%     产为已售或已租。资产一旦售出就不考虑编辑更新其信息了
%   + 使用python命令行界面管理应用
% - 细读需求,可以找出些许名词,它们或许就是潜在的应用里的类。需要表示资
%   产,独栋和公寓是两个类,租和售也不同
% - 主要考虑使用继承技术
% - House, Apartment是Property的子类,Property是父类
% - UML设计图
% [[file:./py/agent.py][源文件]]

% *** 案例3: 计算多边形边长

% - 面向对象分析设计,何时使用oop?
% - 面向对象编程范式下,分析、设计人员
%   + 首先识别出问题领域(domain-driven programming)的对象
%   + 再识别、设计每个对象的属性、方法,及对象间交互联系
%   + 通过以上步骤,为问题领域建模
% - 对象既有数据(属性),又有行为(方法)
%   + 如果仅需要数据,完全可以将数据存储到列表、集合、字典中
%   + 如果仅需要行为,而不关心数据存储,那函数就够了
% - 一般编程思路
%   + 首先会想到使用哪些数据,会把这些数据存入变量
%   + 慢慢地,会发现,总会把相关的一组数据传入某些函数处理
%   + 这时,就应该考虑,把这些变量、函数组合起来,进一步抽象为一个对象
%   + 慢慢的,会发现对象之间的关系,考虑将对象归类,将类构造为继承层次
% - 此时,应该回到分析之出,假设自己,一开始就考虑问题领域里应该有哪些对
%   象。逐渐的,习惯这种思路之后,遇到问题后,首先会考虑,问题领域中有哪
%   些对象,这些对象如何交互实现了各种功能。这就是面向对象分析与设计
% - 举例: 计算多边形边长,两种编程范式对比
%   + 注意代码可读性。
% - 源码: [[file:./py/polygons.py][计算多边形边长]]

% *** 案例4: Modeling a /Document/ class that might be used in a text editor or word processor

% - What objects, funcitons, or properties should it have?
% - ./py/docu

