\chapter{基本数据类型}
\section{字面量}
字面量是某些内置数据类型常量值的记号。记号则指称特定实体。  名-值:名字是记号,值是实体。 例如,1, one, 一是记号,数学概念“1”是实体。  再如,重名的“张三”,记号同,实体不同。 名字和学号,记号不同,实体同。

\section{数字}

数字,亦称数字类型,是字面量。共有三种类型的数字字面量: 整数、浮点数、复数。数字字面量不含符号。例如,-1不是字面量,而是表达式,是由操作符负号-和字面量1组成的表达式。
\subsection{整数}
整数是准确的。
\begin{itemize}
\item 十进制整数 decimal 0,-1,9,123
\item 十六进整制 hexadecimal 0x10, 0xfa
\item 八进制整数 octals 0o35
\item 二进制整数 0b0101
\end{itemize}
\subsection{浮点数}
浮点数是实数的近似(不精确)表示。
\subsection{练习:浮点数}
\inputpython{examples/float.py}{4}{5}
% \paragraph{浮点数的机内存储}
% \paragraph{机内存储释义}
% \paragraph{奇怪的结果}
避免用很小的小数做除数,避免两个相近的浮点数相减、相除,那样会损失精度,造成结果失真。科学计算和数值分析,一般不直接用Python原生功能,而应\pyth{import numpy}使用专用数值计算API。
\subsection{Complex Numbers}
\begin{framed}
\begin{verbatim}
  a = 3 + 4j
  b = 5 + 6j
  c = a + b
\end{verbatim}
\end{framed}
\section{Operator}
%\section{运算符}
称数字为算子,是参与运算的数,被操作的数。称运算符为算符,是运算的方式,操作的动作。
\subsection{Excercise: Operator}
\inputpython{examples/operator.py}{5}{57}
\section{字符串}
字符串(strings),是不可变字符序列(sequence),亦称字符串字面量。
\begin{itemize}
\item 字符串可用单、双、三引号作为界定符,三引号界定的字符串可以跨行
\item 字符串可以连接,\pyth{'Hello, ' + 'world!'}或\pyth{'hello, '  'world!'}
\item \pyth{str()}将值转为字符串string,如,\pyth{str(9)}
\item \pyth{input()}接受用户输入的字符
\item  python3中,字符串以Unicode编码
\end{itemize}

\subsection{字符的转义}
转义符\textbackslash。有一种字符串称为原生字符串(raw stings),原生字符
串前缀r,忽视转义符,原样输出字符串中的字符。原生字符串简化某些繁琐的转
义。
\subsection{练习:转义符}
\inputpython{examples/escape.py}{4}{20}
  

\subsection{课后阅读:字符编码}
\begin{itemize}
\item \href{https://en.wikipedia.org/wiki/ASCII}{ASCII}
\item \href{https://www.unicode.org/standard/WhatIsUnicode.html}{Unicode}
\item \href{https://en.wikipedia.org/wiki/UTF-8}{UTF-8}
\end{itemize}

\section{变量}
\begin{itemize}
\item 变量(variables)是一个指代某值(数字、字符串)的名字
\item 变量最常见操作,称为赋值或绑定,\pyth{x = 3}。变量赋值后,可代替值参与运算: 10 + x
\item 变量优势在于可以在不关心其值的情况下,参与运算
\item python不需要事先声明变量名及类型,绑定即声明,python解释器根据绑定或运算推断变量类型
\item 先绑定后使用
\item 允许多个变量指向,即被绑定到,同一个值
  \begin{itemize}
  \item 变量/值,名/实
  \item 某人张三,可以两名指此人(张三、老张、那人,或化名李四)
  \item 不可一名指两人(重名,歧义,二义性),为消歧,须前缀命名空
    间namespace、域scope。如,1班张三,2班张三
  \item x人行必有我师, y人成虎, 桃园z结义
  \end{itemize}
\item 变量命名规则
  \begin{itemize}
  \item 变量以字母或下划线开头,下划线开头变量在python中有特定含义
  \item 变量中不能有空格及常用标点符号
  \item 不能使用关键字作为变量名
  \item 建议不要使用系统内置的模块名、类型名、函数名及已导入的模块名及
    其成员名,预防重名引起混淆
  \item 变量区分大小写case sensitive
  \end{itemize}
\end{itemize}

\subsection{练习:关键字}
\inputpython{examples/kw.py}{4}{5}

\section{表达式和语句}
表达式 = 算子 + 算符。算子可以是值,也可以是变量。表达式可求值。表达
式2 + 2,值为4。若x = 2,则表达式 x * 3值为6。


语句是python解释器可以执行的合法指令,语句做某些事情,改变某些情况。如
语句x = 2,执行后,变量x被绑定到2这个值上。

目前学习了两个语句:赋值和引入import,此外,注意print在python2中是语句,
在python3中是函数。
\subsection{几个python特有语句}
\paragraph{序列解包赋值}
\begin{python}
  x, y, z = 1, 2, 3
  values = 4, 5, 6
  x, y, z = values
  
  # x, y, z = 1, 2
  # x, y ,z  = 1, 2, 3, 4
\end{python}
\paragraph{链式赋值}
将值绑定到若干变量的捷径 \pyth{x = y = z = 22}
\paragraph{增量赋值}
\pyth{x = 3, x += 1, x *= 3}
\paragraph{空语句}
\pyth{pass}什么也不做,作用是占位。
\subsection{练习:空语句}
\inputpython{examples/pass.py}{4}{18}
第一部分代码无法调试,更不会运行,因python中留白语句块非法。因此,常用
技巧是用空语句pass占留白语句块的位置,让程序运行起来。

\section{函数和模块}
\subsection{函数}
\begin{itemize}
\item 一个函数(functions)就是一小段程序,执行特定功能
  \begin{itemize}
  \item Python自带大量函数可完成许多工作,另有数量庞大的第三方工具包提供了
    更多的函数
  \item 如果这些还不够用,程序员可以编写自己的函数完成特定的工作
  \item Python自带的标准函数称为内置(built-in)函数
  \item 例: 2 ** 3 可用\pyth{pow(2,3)}函数代替
  \end{itemize}
\item 使用函数的方法是调用(calling, invoke)函数,调用函数同时提供参数
  (parameters),得到返回值(return value)
  \begin{itemize}
  \item \pyth{pow(2, 3)},2、3是传递给函数的参数,8是返回值。 
  \item \pyth{print(x)}返回值就是输出内容
  \end{itemize}
\item 由于函数返回值,因此函数是一种表达式,可以将函数与运算符组合,创
  制功能更丰富的表达式。 \pyth{10 + pow(2, 3*5) / 3.0}
\end{itemize}

\paragraph{Python内置函数}
\href{https://docs.python.org/3/library/functions.html}{python内置函数表}
\subsection{模块}
对于非内置函数,使用import语句引入模块,通过语法格式module.function使用特定函数。
\subsection{练习:引入模块}
\inputpython{examples/import.py}{1}{19}
自定义模块可有绝对引入 (absolute import) 和相对引入 (relative import)。
\begin{python}
  import ecommerce.products
  from .database import Database
  from ..database import Database
\end{python}

\section{小结}
须掌握的概念: 数字、字符串、变量、赋值、绑定、运算符、算子、表达式、语句、引入、函数、模块。


