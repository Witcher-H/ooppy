\chapter{基本面向对象技术}
结构化编程,重点在编写函数,处理数据,数据和函数分开定义,松散关联。面向对象编程,重点在创建对象。对象既包含数据又包含处理数据函数,二者定义放入同一个数据结构中,密不可分。这个数据结构称为\emph{类}。程序运行时,依据类中的定义,创建对象。

面向对象程序设计技术源于对隐藏信息的需要。结构化编程与面向对象编程,如海大和吉大图书馆,分别采用开放数据与控制访问制度。书就像数据,控制访问情况下,只能通过馆员借还书,绝对不允许自己动手。面向对象程序中,对象的数据称为\emph{属性},对象的函数称为\emph{方法}。这样的程序中,数据仿佛被隐藏和保护起来,访问受到严格控制,只能通过特定函数读取、修改数据。这种隐藏和保护数据术语称为\emph{信息封装}。

\section{一切皆对象}
Python支持面向对象编程,将数据和操作数据的函数抽象为一整体,用术语\emph{对象}来刻画这个整体。每个对象有唯一的标识、属于某种数据类型,有值。Python中所有事物都是对象。列表是对象,42是对象,模块是对象,函数是对象。
\begin{quotation}
  One of my goals for Python was to make it so that all objects were ``first class."  By this, I meant that I wanted all objects that could be named in the language(e.g., integers, strings, functions, classes, modules, methods, and so on) to have equal status. That is, they can be assigned to variables, placed in lists, stored in dictionaryes, passed as argumetns, and so forth.''
  --Guido van Rossum(Blog, The History of Python, February 27, 2009)
\end{quotation}
\subsection{对象标识}
\pyth{id(object)}显示一个对象的标识。
\begin{python}
def foo(): pass

type(foo), id(foo)
# (<class 'function'>, 38110760)   #  function is an object
type(foo.__code__)
id(foo.__code__)
# (<class 'code'>, 38111680)  #  code is an object
\end{python}
\subsection{对象的数据类型}
对象的数据类型定义了对象数据的取值范围,同时定义了对象可以完成的操作。数据类型本身也是对象,它的数据类型就是自身。
\begin{python}
type(42)
#  <class 'int'>
type(type(42))
#  <class 'type'>
type(type(type(42)))
#  <class 'type'>
def f(x):
    return x+1
type(f)
import math
type(math)
\end{python}

\section{类}
类是用户自定义对象的数据类型,是创建对象的模板。定义类语法如下。
\begin{framed}
\begin{verbatim}
  class 类名(父类1, 父类2...):
      """类说明"""
      def __init__(self, arg1, arg2...):
          """方法说明"""
          构造器方法体

      def method1(self, arg1, arg2...):
          """方法说明"""
          方法1语句

      ……
\end{verbatim}
\end{framed}
下面例子用类的语法,创建一个学生类。
\begin{python}
class Student(Human): 
    def __init__(self, name='Adam', sex='Male', age=18):
        self.name = name
        self.sex = sex
        self.age = age

    def ChangeName(self, new_name):
        self.name = name
         
s = Student('Mary', 'Female, 19)
type(s)
type(type(s))
\end{python}
上面例子中,按Python语法,创造了自定义数据类型学生类。术语“类”和“数据类型”是同一个概念的两个名字。为防混淆,一般称内置数据类型为数据类型,称自定义数据类型为类。类合数据和操作为一,提供创造新数据类型的机制。创建一个类,就是创建了某种对象的新数据类型。类有属性,描述状态,代表类的数据,表示类的静态方面;类有方法,修改状态,代表了类的行为,表示类的动态方面。数据抽象主要思想是鼓励程序定义自己的数据类型。
\subsection{实例}
实例(instance)是对象的同义语。 “实例是某数据类型的对象”。例如,42是int数据类型的实例,即42是一个整型对象。上例中,s是学生类的实例,是学生对象,用来表示一个具体的学生Mary,其数据类型是学生类。

回想下,我们已在前面学习中,接触到类和对象了。比如,列表是类,具体的列表是列表实例,而列表方法,正是该类的方法。
\subsection{构造器和析构器}
\pyth{__init__()}称为\emph{构造器},该方法用于创造一个实例化的类——对象。该方法是创建对象后,第一个调用的方法。

\pyth{__del__()}称为\emph{析构器},该方法用于摧毁对象。由于Python有支持垃圾收集技术(garbage collection),析构器几乎很少使用。
\subsection{增加属性和方法}
属性描述类的性质。类的不同实例初始化为对象后,有各自的属性值,对象属性值描述每个对象的状态。属性与变量一样,赋值即定义。属性赋值语句:\pyth{object.attribute = value}。习惯上,在构造函数内,声明对象属性并赋值。

方法描述类的行为,指明类的对象能做哪些事情。对象方法做的主要事情,就是改变对象的属性值,即改变对象的状态,完成从一个状态到另一个状态的转移。方法和函数的区别在于,方法必须依附于某个对象。因此,方法的所有参数中,第一个参数必须是调用该方法的对象,即该方法依附的对象。习惯上,这个参数命名为self。

一般,属性指对象的属性,方法指对象的方法。后面还要学习类属性、类方法,它们与对象的属性和方法不同。
\subsection{读取属性和调用方法}
读取属性语法为\pyth{object.attribute},调用方法语法为\pyth{object.method(arg1,arg2...)}。有同学可能会注意到,调用方法时,参数列表中第一个参数并没有提供self对象。这如何解释?可以这样理解\pyth{mehtod(object, arg1, arg2)}。这也再次说明了方法与函数的差别。

下面这个例子,展示了构造器、析构器、属性、方法的用法。读懂这个例子后,做后面的练习。
\inputpython{\ooeg/init.py}{1}{30}
\subsection{练习:创造类}  
二维平面中,视点为一个实体,其属性为直角坐标系下横纵坐标,如(0, 0)、(x, y)。与点有关的操作有,计算该点到原点距离,计算该点到某点的两点间距离,计算两点的中点坐标,或该点是否在某个矩形或圆形范围内等等。

尝试自定义类Point,代表二维平面上的点Point,该类有两个属性,分别代表横纵坐标。方法有:构造器;移动,修改点的坐标;重置,将点坐标设置为原点;计算距离,如果已知另一个点,则可计算其与本点的欧式距离。
\inputpython{\ooeg/Point.py}{1}{45}  

\section{对象特征}
面向对象语言支持一些强大的技术。这些技术也成为面向对象程序最显著的特征。结构化语言不支持这些特征。
\subsection{信息封装}
信息封装是一种信息访问保护机制,用于控制对属性和方法的读取和修改。属性和方法的封装类型一般分三个层次。
\begin{table}
  \centering
  \begin{tabular}{rcl}
    \toprule
    属性和方法名 & 类型 & 类型含义 \\
    \midrule
    name        & 公开  & 常规命名,供所有类自由访问、调用,没有限制 \\
    \_name      & 保护  & 前缀下划线,供该类及其子类使用,其他类不应该访问 \\
    \_\_name    & 私有  & 前缀双下划线,仅供该类使用,其他类绝不应该访问 \\
    \bottomrule
  \end{tabular}
  \caption{属性封装类型}
\end{table}
从技术角度看,Python在语言层面没有提供严格访问保护机制,所有属性和方法本质上都是公开的。 程序中,如果该属性供内部使用,不希望其他类访问,应在文档中注明,这是一个内部方法。或者,按大多数Python程序员的习惯做法,修饰名称(name mangling)。这个术语指这样一种技术,通过给属性或方法名,加一个下划线前缀,来标记这是一个应该隐藏的属性。更强烈的标记是前缀双下划线。事实上,访问双下划线属性也是可以的,只不过写法上费点事。如此费事,岂不正是程序员强烈提醒用户不要使用此属性或方法。另外,双下划线,是许多特名(special names)的标志,比如构造器。因此,程序员给属性、方法起名,为了标识信息隐藏,一个下划线足够,尽量不要使用双下划线。
\inputpython{\ooeg/dunder.py}{1}{18}

\subsection{继承}
继承是面向对象程序设计最显著的特征,是重用代码的一种方法。甚至有人说:“如果没有继承,面向对象程序设计几乎不值一提。”继承允许在两个乃至多个类之间建立表示上位类和下位类的继承(is a)关系。上位类称为父类、超类,下位类称为子类。由此,在类之间建立了层次关系。这与自然界中生物间继承关系很像。采用继承技术的主要动力是重用代码,即希望不要在两个相似的类中,重复撰写相同的代码,而是让子类继承父类的属性和方法,达到复用目的。一般将共性的属性、方法放到父类中,将个性的属性、方法放到子类中,则子类通过继承同时拥有共性和个性的属性和方法。

Python中每个类都使用了继承关系,所有类都继承自一个特殊类object。这个类几乎没有实用功能。但它提供了统一的视角,看待所有Python类。大家都是object类的子类。继承分单继承和多继承,Python支持多继承。本节只讲单继承,多继承留到下一章讲。
\inputpython{\ooeg/sglinheri.py}{1}{49}
可以看出,人可以做自己的事情,即调用自己的方法,教师、学生可以做各自事情。此外,人和学生、人和教师通过继承建立种属(is a)关系。学生是人,当然可以做人可以做的事情。教师是人,可以做人做的事情。而人未必是学生,人不能做学生做的事情,人未必是教师,人不能做教师做的事情。学生不是教师,教师也不是学生,互相不能做对方的事情。这就是继承的优势,子类继承父亲的属性和行为,同时增加自己特有的属性和行为。教师、学生都复用了人的属性和方法,而不必在自己类内将这些方法重写一遍。。
\subsubsection{覆盖}
覆盖(override)技术,指子类用同名,内容不同的新方法替换父类中的方法。子类方法覆盖、遮蔽同名的父类方法。

例子,覆盖父类方法。
\inputpython{examples/ooeg/override1.py}{1}{14}
覆盖是继承机制中很重要的一项技术,对于构造函数尤其重要。子类通过覆盖,继承了父类方法名字,但改变了父类的行为。上例是完全改变父类行为。学生例子中,扩展了父类行为。学生继承人的属性和方法,但使用覆盖技术,子类的\pyth{__init__()}方法,覆盖了父类的\pyth{__init__()}方法,表示除父类的行为外,子类还要增加行为,形成自己的行为。调用\pyth{Person.__init__()},表示学生全盘继承人的构造器的全部行为,即设置名、姓,同时,学生还在此行为基础上增加自己的行为,即设置学号。

覆盖技术,避免代码冗余,不必在子类内重复写父类的代码。覆盖技术,还有助于维护代码。如果想修改代码,为人增加一个属性\pyth{self.middle_name},我们只须在\pyth{Person}类的构造器增加即可,学生、教师会自动继承新增加的属性。

这里我们直接用了父类的名字\pyth{Person},更好的办法是用\pyth{super().__init__(arg1, arg2)}指代父类。注意super用法的参数列表中,不再包含self。实际上这个方法可看成是\pyth{__init__(super(), arg1, arg2)},因为要调用父类的构造器,第一参数位置应放父类,如果放self,self此时指代子类,会报错。
\inputpython{\ooeg/override2.py}{1}{42}
\subsection{多态}
多态(polymorphism)技术,指当不知道调用方法的子类是哪个时,系统会自动根据子类所属类别,执行相应的方法,产生不同的行为。
例子,多态
\inputpython{\ooeg/simplepoly.py}{1}{65}

\section{练习:模拟音频播放器}
假设用程序播放音频。一个播放器对象(media player object)创建AudioFile对象,执行play方法。play()方法负责解码音频文件,并播放,其行为可简单描述为audio-file.play()。但解码音频文件,需要根据不同的文件使用不同的算法,wav,mp3,wma,ogg,显然不能用同一种算法解码。如果,要为每种文件定义一个play()方法,那得定义很多少个方法,而且 随着技术发展,出现新的音频文件格式怎么办,定义多少个方法够用呢?使用继承和多态可以简化设计。每种格式用AudioFile的一个子类表示,如WavFile, MP3File。每个子类都有自己的play()方法。但该方法根据不同子类实现不同的解码算法和播放功能。而播放器对象永远不必知道AudioFile指的是哪个子类,播放器只需调用play()方法,让对象的多态机制处理具体的解码播放即可。

本练习涵盖继承、覆盖、多态技术。
\inputpython{\ooeg/poly.py}{1}{73}
