\chapter{控制流:选择与循环}
\section{python代码风格}
Python有几个突出特点:
\begin{itemize}
  \item 不使用分号(;)做语句结束符
  \item 不使用成对花括号({})作为边界符, 而是使用缩进(Indentation),一
    般是4个空格(而不是1个tab),界定语句块。
  \item 不要忽略冒号(:)
  \item \#为注释符
\end{itemize}
\inputpython{examples/style.py}{1}{11}
写程序就是填空,见下例。
\begin{verbatim}
def 函数名(参数列表):
    """
    docstring: 函数说明,说明函数功能,使用何种算法,参数个数、类型,返回值等。
    """
    #  注释
    代码块(函数体)
    1. 输入
    2. 处理
    3. 输出
    return 返回值
\end{verbatim}
\subsection{Reading}
\href{https://www.python.org/dev/peps/pep-0008/}{Python推荐编码风格}
\section{语句块}
块(block)是一组语句,作为一个整体。当条件表达式值为“真”时执行,或在循
环中执行若干次。冒号表明语句块起点,缩进界定语句块,所有连续、同缩进格
式的语句属于同一个语句块。
\section{条件表达式}
Python中,单个常量、变量、任意合法表达式、或由简单条件表达式与运算符组
合而成的复杂条件表达式都可视为条件表达式。这些表达式的值就是条件表达式
的值。

条件表达式有且仅有两个值,任意条件表达式值必居其一:
\begin{itemize}
\item ``假'': Fasle, 0, 0.0, 0j, None, 空列表[],空元组(),空集合,空字
  典{},空字符串'',空range或其他空iterable对象
\item ``真'': 其他情形
\end{itemize}
程序可以根据条件表达式的值,决定是否执行某个语句块。
\subsection{条件表达式的组合}
例如,\pyth{((a > 10) and < (a < 20))}。 组合后,其值依然“真”、“假”必居其一。
\subsection{辨析 == 和 is}
\begin{description}
\item{==} 相等运算符 the euqality operator
\item{is} 同一运算符 the identity operator, 似乎与==相类,注意区别
\end{description}
\subsection{练习:相等和同一}
\inputpython{examples/identity.py}{5}{13}
两个运算符的区别在于:
\begin{itemize}
\item is 判断的是同一性(identity),而不是(值)相等
  性(equality)。 x和y被绑定到同一列表,z被绑定到另一个列表,该列表恰
  与xy列表值相等。 尽管值等,但z与xy不是同一对象
\item xy与z代表的列表相等equal, 但不同non-identical
\item 用法: 用==判断两个对象值是否相等,用is判断两个对象是否同
  一identical
\item 在CPython实现中,python的identical指x、y存储了同一个对象的内存地
  址
\end{itemize}
\section{分支结构}
\subsection{单分支}
\begin{framed}
\begin{verbatim}
  if 表达式:
      语句块
\end{verbatim}
\end{framed}
\subsection{练习:单分支}
\inputpython{examples/if.py}{1}{13}
\section{双分支}
\begin{framed}
\begin{verbatim}
if 表达式:
    语句块1
else:  #   else clauses
    语句块2
\end{verbatim}
\end{framed}
\subsection{练习:双分支}
\inputpython{examples/ifelse.py}{1}{17}
\section{多分支}
Python 不支持switch case关键字,多分支结构的功能用\pyth{if...elif...else}完成。
\begin{framed}
\begin{verbatim}
if 表达式1:
    语句块1
elif 表达式2:  #   elif clauses
    语句块2
elif 表达式3:
    :
    :
    :
else:
    语句块n
\end{verbatim}
\end{framed}
\subsection{练习:多分支}
\inputpython{examples/elif.py}{1}{7}
\section{Nested Switch}
% \section{分支嵌套}
使用嵌套结构一定严格、清晰控制不同级别语句块的缩进量,缩进量决定着语句
块的从属关系,影响着程序的执行路径和逻辑功能。
\begin{framed}
\begin{verbatim}
if 表达式1:
    语句块1
    if 表达式2:
        语句块2
    else:
        语句块3
else:
    if 表达式4:
        语句块4
\end{verbatim}
\end{framed}
\subsection{Exercise: Nested Switch}
\inputpython{examples/nestedswitch.py}{1}{11}
\section{Asser}
assert逻辑上等价于if not condition,用于测试。assert语句用法为,断言某表达式(通常是布尔表达式)为真,并附言解释为何如此。当该表达式为假时,assert爆出异常,显示附言,终止程序。当表达式为真,什么都不发生,表明该断言测试的值在合理范围内,没出错。
\subsection{Exercise: Asser}
\inputpython{examples/assert.py}{1}{5}
\subsection{Reading}
\begin{itemize}
\item \href{https://docs.pytest.org/en/latest/assert.html}{The writing and reporting of assertions in tests}
\item \href{https://cacm.acm.org/magazines/2014/2/171689-mars-code/abstract}{mars code}
\item \href{https://github.com/pyclectic/pyassert}{pyassert}
\end{itemize}
\section{Loop}
\subsection{while}
\subsubsection{练习:while}
\inputpython{examples/while.py}{1}{11}
\subsection{for}
\subsubsection{练习:for}
Python风格for循环: \pyth{for var in iterable/range:}
\inputpython{examples/for.py}{1}{12}
\subsection{break, continue}
\begin{itemize}
\item break语句,退出当前循环,提前结束整个循环
\item continue语句,终止本次循环,忽略continue之后的所有语句,直接回到
  循环顶端,提前结束本次循环,进入下一次循环
\end{itemize}
\subsubsection{练习:break, continue}
\inputpython{examples/break.py}{1}{29}
\subsection{循环可带else子句}
语句可独立存在,子句不能独立存在,必须依附于某些语句,才起作用。
\subsubsection{练习:循环子句}
\inputpython{examples/loopelse.py}{1}{20}

\section{课堂练习}
阅读并理解代码。
\begin{framed}
\begin{verbatim}
  # 1. 计算1 + 2 + 3 + ... + 100的值
  s = 0
  for i in range(1, 101):
      s = s + i
  print(s)
  print(sum(range(1, 101)))

  # 2. 求1-100间能同时被7整除,不能被5整除的所有整数。
  # 那同时能被7和5整除的整数呢?
  for i in range(1, 101):
      if ((i % 7) == 0) and ((i % 5) != 0):
          print(i)

  # 3. 水仙花数输出“水仙花数”,所谓水仙花数,指3位十进制数,其各位数字立方
  # 之和等于该数,例如 153 = 1^3 + 5^3 + 3^3
  for i in range(100, 1000):
      ge = i % 10
      shi  = i // 10 % 10
      bai = i // 100
      if (ge**3 + shi**3 + bai**3)  == i:
          print(i)

  # 4. 求平均
  score = [70, 90, 78, 85, 97, 94, 65, 80]
  s = 0
  for i in score:
      s += i
  print(s / len(score))

  print(sum(score) / len(score))

  # 5. 输出99乘法表
  for i in range(1, 10):
      for j in range(1,10):
          if (j<=i):  # 调整格式
              print(j, '*', i, '=', i*j, ' ', end='')
      print()  # 输出空行

  # 6. 求200以内能被17除的最大正整数
  for i in range(200, 0, -1):
      if (i % 17) == 0:
          print(i)
          break

  # 7. 判断一个数是否为素数
  # 遍历N能否能被从2到sqrt(N)之间的素数整除。若不能则为素数。
  import math
  
  n = int(input('Input an integer:'))
  m = math.ceil(math.sqrt(n) + 1)

  for i in range(2, m):
      if ((n % i) == 0) and (i < n):
          print('No')
          break
  # NOTE THE POSITION OF THe ELSE
  else:  
      print('Yes')

  # 8. 鸡兔同笼问题, 设共有鸡兔30只,查有90只脚,问鸡兔各多少只?
  for chick in range(0, 31):
      if ((2 * chick) + ((30 - chick) * 4)) == 90:
          print('Chicks:', chick, 'Rabits:', (30 - chick))
\end{verbatim}
\end{framed}

\section{课后作业}
\begin{itemize}
\item 编写程序,运行后用户输入4为整数年份,判断是否为闰年。 判断闰年算
  法如下:如果年份能被400整除,是闰年; 如果年份能被4整除,但不能被100整
  除,也是闰年
\item 编写程序,用户从键盘输入小于1000的整数,程序对其进行因式分解。
  例: 10 = 2 × 5, 60 = 2 × 2 × 3 × 5
\item 编写程序,实现分段函数计算,见表\Cref{segfunc}。
\end{itemize}

\begin{table}
  \centering
  \begin{tabular}{cc}
    \toprule
    x           & y     \\
    \midrule
    x < 0       & 0     \\
    0 $\le$ x < 5  & x     \\
    5 $\le$ x < 10 & 3x-5  \\
    10$\le$ x < 20 & 0.5x-2\\
    20$\le$ x      & 0     \\
    \bottomrule
  \end{tabular}
  \caption{分段函数}%Segment function}
  \label{segfunc}
\end{table}
\section{阅读:代码可读性}
\begin{itemize} 
\item \href{https://zhuanlan.zhihu.com/p/22334966}{代码可读性提升指南}
\item \href{http://blog.jobbole.com/73791/}{请优先提高代码的可读性}
\item \url{https://stackoverflow.com/questions/550861/improving-code-readability}
\item \url{https://blog.codinghorror.com/code-smells/}
\item \url{https://en.wikipedia.org/wiki/Readability#Popular_readability_formulas}
\item 《代码大全(第二版)》
\item 《代码整洁之道》
\item 《重构: 改善既有代码的设计》
\item 《编写可读代码的艺术》
\end{itemize}
