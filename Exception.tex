\chapter{异常}
程序应区分正常情况事件和异常情况事件,异常事件可能是错误(errors),还可能是发生了预想之外的情况。为处理异常情况,初步想法是用条件表达式判断各种可能的异常情况。这种处理方式效率低,缺乏变通,很难覆盖所有异常情况,导致程序频出意外。Python提供了一套处理异常的方法,创建、弹出、并处理异常。
\section{异常对象}
Python用异常对象(exception objects)代表异常情况。出现错误,异常对象爆出异常(raises an exception)。如果该异常对象没有得到处理,程序终止,返回回溯(traceback)或错误提示。回溯就是异常的输出结果,阅读时,须从下向上读。其理念是当出错时,自动弹出异常。

Python有多少异常类,其互相之间关系如何?请看
\href{https://docs.python.org/3/library/exceptions.html#exception-hierarchy}{python异常类层次关系}
\section{爆出异常}
使用raise和一个异常类或异常对象,即可爆出异常。

\begin{python}
  raise Exception
  raise Exception('hyperdrive overload')  # raise exception with message
  raise ArithmeticError
\end{python}
\section{自定义异常}
Python自定义了若干异常类,描述了很大一部分异常情况。
\begin{python}
import exceptions

dir(exceptions)  # Check built-in Exception
\end{python}
尽管有如此丰富的内置异常类,然还有不敷用之时。这就需要程序员自定义异常
类。定义异常类和其他类定义方法一样,但要求自定义异常类必须是Exception子
类。

定义异常的语法为
\begin{verbatim}
class 自定义异常名(Exception):
    类定义语句
\end{verbatim}
% \section{Trap Exception}
\section{捕捉异常}
处理异常称为捕捉异常。Python使用\pyth{try...except...finally}子句结构捕
捉异常。
\subsection{基本结构}
\begin{framed}
\begin{verbatim}
try:
    try块    受监控、可能出意外的语句,
except Exception1:    异常情况1,exception1
    except 块1          处理异常情况1的语句
except Exception2:       异常情况2,exception2
    except 块2          处理异常情况2的语句
...                      直至n
finally子句中的语句块,无论是否发生异常都会执行,
常用来做一些清理工作以及释放try语句中申请占用的资源
finally:
    finally块            无论如何都会执行的语句
\end{verbatim}
\end{framed}
例如:
\begin{python}
x  = 10
y = 0
print(x/y) # error without exception

try:
    x = 10
    y = 0
exception ZeroDivisioinError:
    print("The second number can't be zero!")
\end{python}
比较一下,用if条件表达式来判断异常,在上例情况是可行的。但如果有十种异常情况,就需要至少十个if条件表达式,程序会很乱。而使用try...except结构会十分清晰。
\subsection{多个except子句}
\begin{python}
try:
    x = 10
    y = 0
    print(x/y)
except ZeroDivisionError:
    print("The second number can't be zero!")
except TypeError:
    print("That wasn't a number, was it?")
except NameError:
    print("No such name!")

try:  # catch multiple exceptions
    x = 10
    y = 0
    print(x/y)
except (ZeroDivisionError, TypeError, NameError):
    print("There is definitly something wrong!")
\end{python}

\subsection{访问异常对象}
可以在except子句中访问异常对象。
\begin{python}
try:
    x = 10
    y = 0
    print(x/y)
except (ZeroDivisionError, TypeError) as e:
    print(e)
\end{python}

尽管如此多异常类供使用,但有些没考虑到的异常依然能穿透try...except语句,
突然冒出来。为防止有漏网的异常,人们想捕捉所有异常,那么仅写except子句,
而不要写异常类名。
\begin{python}
try:
    x = 10
    y = 0
    print(x/y)
except:
    print("Something wrong happened...")
\end{python}
但是,捕捉所有异常并不是一个好方法,仅是推脱。好办法是预先考虑尽可能多
异常,安全的处理它们,而不是一股脑推给except。
\subsection{else子句}
\begin{python}
try:
    x = 10
    y = 0
    print(x/y)
except NameError:
    print("Unknown variable")
except ZeroDivisioinError:
    pritn("Zero division")
else:
    print("That went well!")  # no exception, do something
finally:
    print("Cleaning up.")     # The finally clause will be executed any way
\end{python}

\subsection{例子}
\pyth{auth}

% ** 调试和测试

% - 了解、自学debug and test
%   + debug: pdb
%   + test: unittest, pytest etc.

