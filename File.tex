\chapter{文件}
为长期保存数据,须将数据以文件形式存储到磁盘中。文件分为文本文件、二进制文件。文件操作流程如下。
\begin{enumerate}
\item 打开文件并创建文件对象
\item 通过文件对象对文件内容进行读取、写入、修改、删除
\item 保存文件,关闭文件对象
\end{enumerate}

\section{打开文件}
\pyth{open(name[, mode[, buffering]])}函数打开文件。扩展巴克斯范式(Extended Backus-Naur form)用于描述函数。其中,name是必要(madatory)参数,mode和buffering是可选(optional)参数,该函数返回一个文件对象。
\begin{python}
f = open(r'c:\text\somefile.txt')
\end{python}
打开后,f文件对象代表该文件。
\subsection{打开模式}
\begin{table}
\centering
\begin{tabular}{cc}
  \toprule
  值  & 说明 \\
  \midrule
  'r' & 只读 \\
  'w' & 写入 \\
  'a' & 追加 \\
  'b' & 二进制模式(可组合) \\
  '+' & 读写模式(可组合) \\
  \bottomrule
\end{tabular}
\caption{文件打开模式}
\end{table}
一般,Python默认处理文本文件,如处理二进制文件需要明确指定二进制模式。缓存标志位,0表示不缓存,1表示使用缓存。
\section{文件操作}
\subsection{读写文件内容}
上下文(context)管理关键字with自动完成管理资源,不论何种原因跳出with块,总能保证文件正确关闭。操作文件时,应用这种方式打开文件。
\begin{python}
f = open('somefile.txt', 'w')
f.write('1')
f.write('2')
f.close()

f. = open('somefile.txt', 'r')
f.read(4)
f.read()

# recommend syntax, using context
with open('somefile.txt', 'w') as f:
    f.write('Hello,')
    f.write('World!')
    
\end{python}
\subsection{按行读写文件}
\begin{python}
with open('sample.txt','r') as f:
    while True:
        line = f.readline()
        if line == '':
            breal
            print(line, end='')
# sample.txt
# Welcome to this file
# There is nothing here except
# this strange line
f = open(r'c:\text\sample.txt')
f.read(7)
f.read(4)
f.close()

f = open(r'c:\text\sample.txt')
print(f.read())
f.close()

f = open(r'c:\text\sample.txt')
for i range(3):
    print(str(i) + ":" + f.readline())
f.close()

import pprint
pprint.pprint(open(r'c:\text\sample.txt').readlines()) # file object is being closed automatically.

# this
# is no
# haiku
lines[1] = "isn't a\n"
f = oepn(r'c:\text\sample.txt', 'w')
f.writelines(lines)
f.close()

# one char a time
def process(string):
    print('Processing: ', string)

f = open(filename)
while True:
   char = f.read(1)
   if not char: 
        break
   process(char)
f.close()

# one line a time
f = open(filename)
while True:
   line = f.readline()
   if not line: 
        break
   process(line)
f.close()

f = open(filename)
for char in f.read():
    process(char)
f.close()

f = open(filename)
for line in f.readlines():
    process(line)
f.close()

# Lazy Line Iteration with fileinput
# lazy line iteration
import fileinput
for line in fileinput.input(filename):
    process(line)

# file objects are iterable
f = open(filename)
for line in f:
    process(line)
f.close

# Iterating over a file without storing the file object in a variable
for line in open(filename):
    process(line)

\end{python}
\section{操作二进制文件和目录}
\subsection{二进制文件}
Python使用pickle模块,操作二进制文件。pickle是常用且速度快的二进制文件序列化模块。使用pickle进行对象序列化和二进制文件读写。
\begin{verbatim}
import pickle

dir(pickle)  #  查看pickle模块属性
#  定义数据
f = open('sample_pickle.dat', 'wb')
n = 7
i = 13000000
a = 99.056
s = '中华人民共和国 abc 123'
lst = [[1, 2, 3], [40, 50, 60], [7, 8, 9]]
tpl = (-5, 10, 8)
clctn = {4, 5, 6}
dct = {'a': 'apple', 'b': 'banana', 'g': 'grape', 'o': 'orange'}

#  将数据写入二进制文件
try:
    pickle.dump(n, f)  #  转储
    pickle.dump(i, f)
    pickle.dump(a, f)
    pickle.dump(s, f)
    pickle.dump(lst, f)
    pickle.dump(tpl, f)
    pickle.dump(clctn, f)
    pickle.dump(dct, f)
except:
    print('Writing file exception')
finally:
    f.close()

#  从二进制文件读入数据,并输出到屏幕
try:
    f2 = open('sample_pickle.dat', 'rb')
    n = pickle.load(f2)
    i = 0
    while i<n:
        x = pickle.load(f2)
        print(x)
        i += 1
except Exception as e:
    print(e)
finally:
    f2.close()
\end{verbatim}
\subsection{操作目录}
处理文件路径,要使用\pyth{os.path}模块中的对象和方法。该模块提供大量用于路径判断、切分、连接及文件夹遍历方法

例: 查找特定后缀文件
\begin{python}
import os
# dir(os.path)
print([fname for fname in os.listdir(os.getcwd()) if os.path.isfile(fname) and fname.endswith('.org')])
\end{python}
例: 遍历文件夹
\begin{python}
import os

for root, dirs, files in os.walk(".", topdown=False):
    for name in files:
        #  print(os.path.join(root, name))
        pass
    for name in dirs:
        #  print(os.path.join(root, name))
        pass
\end{python}

\subsection{练习}
简单爬取网页、解析文本程序。
\begin{python}
from urllib.request import urlopen
from BeautifulSoup import BeautifulSoup  # install BeautifulSoup

html = urlopen('http://python.org/community/jobs').read()
# print(html)
soup = BeautifulSoup(text)

jobs = set()
for header in soup('h3'):
    links = header('a', 'reference')
    if not links: 
        continue
    link = links[0]
    jobs.add('%s (%s)' % (link.string, link['href']))

print('\n'.join(sorted(jobs, key=lambda s: s.lower())))
\end{python}
