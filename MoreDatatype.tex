\chapter{高级数据类型}

\section{内置数据类型}
数据结构,是将数据元素按特定方式组织成为有结构(数据元素之间有明确关系)的数据集合。程序设计语言中的数据结构的类型一般称为数据类型许多数据类型与对象的集合有关。具体来说,数据类型的值就是一组对象的集合,所有操作都是关于添加、修改、删除或是访问集合中的对象。\href{https://docs.python.org/3/reference/datamodel.html}{Python数据模型}详尽说明了Python支持的数据类型,这些数据类型称为内置数据类型。
Python主要内置数据类型有:
\begin{itemize}
\item 数字,含整数、浮点数、复数
\item 序列
\item 映射
\item 类
\item 实例
\item 异常
\item 其他不常用的类型,如集合
\end{itemize}

Python最常用的数据类型是序列。序列中的每个数据元素都附着一个整数,标示该数据元素在序列中的位置,称该整数为索引,索引从0开始计数。程序员通过索引可以访问并使用该数据元素。Python还经常使用的一种数据类型叫作映射。映射不使用整数索引访问数据元素,而是给映射中的数据元素附着一个名字,称该名字为键。程序员通过键可以访问并使用映射中的数据元素。序列有多种,典型代表是列表和元组,映射只有一种,字典。

Python3的\href{https://docs.python.org/3/library/stdtypes.html#sequence-types-list-tuple-range}{几种基本内置序列类型}:
\begin{itemize}
\item 列表 lists
\item 元组 tuples
\item 字符串 strings
\item 范围对象 range objects
\end{itemize}
列表适用范围广,几乎适用于所有情况,元组适用范围要窄一些。

\section{列表及列表函数}
当使用组合数据时,序列即派上用场。列表是序列最典型的代表。列表的创建、访问、修改、删除,均可通过列表函数完成。其语法为:\pyth{function(list)}。
\begin{python}
zs = ['ZhangSan', 20]
ls = ['LiSi', 19]
database = [zs, ls]
database
\end{python}
\subsection{创建列表}
\begin{itemize}
\item 列表以[]为界定符,以逗号为分隔符
\item 使用=将变量赋值给列表即可创建列表对象。其执行顺序是先创建列表对象,再将变量绑定到列表
\item 使用\pyth{list()}函数可将其他数据类型转换为列表类型
\end{itemize}
\subsubsection{示例:创建列表}
\inputpython{examples/createlist.py}{1}{25}
\subsection{查找列表元素}
\subsubsection{索引}
列表中所有元素的索引,从0起始编号。索引可以取负值。
\begin{python}
greeting = 'hello'
greeting[0]
greeting[-1]

a_list = [3, 4, 5, 7]
a_list[4] # out of range
\end{python}
\subsubsection{切片}
索引用以访问单个元素,切片用以访问连续多个元素。
\begin{itemize}
\item 一般使用两个索引,中间一个冒号
\item 第一个索引所指的元素在,而第二个索引所指的元素不在访问范围内
\item 第一个索引位置要在第二个索引前,颠倒了,只能得到空列表
\item 切片可以使用3个索引数字,第三个称为步长,number[0:10:2]
\end{itemize}
\subsubsection{示例:切片访问列表元素}
\inputpython{examples/lstslicing.py}{1}{67}
\subsection{修改列表元素}
列表具有可改变(mutable)的性质, 即可修改其内部数据元素。并不是每种序列都具有可改变的性质。后面讲的元组,也是序列之一种,但其具有不可改变(immutable)的性质,其内部元素是固定的,不可修改。
\subsubsection{增加}
用+可以拼接序列。
\subsubsection{示例:增加列表元素}
\inputpython{examples/lstadding.py}{1}{3}
\subsubsection{倍增}
用*可以倍增拼接序列。
\subsubsection{示例:倍增列表元素}
\inputpython{examples/lstmultiply.py}{1}{20}
\subsubsection{赋值修改}
给某个索引位置赋值,修改列表一个元素。赋值给超范围索引,则报错。
\begin{python}
x = [1, 1, 1]
x[1] = 2
x

y = []
x[42] = 'Foobar'  #  IndexError
\end{python}
\subsubsection{切片赋值}
切片赋值是给若干个连续位置赋值,修改列表多个元素。
\begin{python}
name = list('perl')
name
name[2:] = list('ar')
name

a = list('Perl')
a[1:] = list('ython')
a

# equal to insert 
n = [1, 5]
n[1:1] = [2, 3, 4]
n

# equal to delete
n[1:4] = [] # del n[1:4]
n
\end{python}
\subsection{删除元素}
删除元素后,该元素消失,列表的长度随之改变。
\begin{python}
len(x)
del x[1]
x
len(x)
\end{python}
\subsection{归属}
使用关键字\pyth{in}检查元素值是否属于列表,结果返回布尔值。
\subsubsection{示例:元素归属}
\inputpython{examples/lstmember.py}{1}{22}
\subsection{其他常用列表函数}
\begin{python}
numbers = [100, 34, 678]
len(numbers)
max(numbers)
min(numbers)
zip()
enumerate()
reversed(seq)
sorted(seq)
tuple(seq)
\end{python}

\section{列表方法}
方法是一种特殊函数,是紧密依附(绑定)于某个对象的函数。调用方法,语法上有特定的要求。调用方法的语法为:\pyth{object.method(arguments)},必须指明其依附的对象。列表方法依附于列表对象,使用列表方法必须指明该方法操作的列表。使用列表方法能访问和修改列表内容,完成的任务与列表函数相似。
\subsection{index}
\pyth{index()}方法获取指定元素首次出现的索引,若该元素不存在于列表内,则报错。其功能是以值求索引,与以索引求值互为逆操作。
\begin{python}
knights = ['we', 'are', 'the', 'knights', 'who', 'we','say', 'ni']
knights.index('we')
knights.index('herring')
knights.index('who')

knights[4] # see the difference
\end{python}
\subsection{count}
\pyth{count()}计算元素在列表中出现的总次数,即元素频次。
\begin{python}
['to', 'be', 'or', 'not', 'to', 'be'].count('to)

x = [[1,2], 1, 1, [2, 1, [1, 2]]]
x.count(1)
x.count([1, 2])
\end{python}
\subsection{insert}
\pyth{insert()}将一个元素增加到列表指定位置。
\begin{python}
numbers = [1,2,3,4,5,6,7]
numbers.insert(3, 'four')
numbers

# slicing could be used to do what insert does
numbers[3:3] = ['Five']
numbers
\end{python}
\subsection{append}
\pyth{append()}方法也是向列表末尾增加元素,其操作是所谓“就地改变(change the list in place)”。其含义是,直接修改原列表对象。优点是快、节省空间,缺点是丢失了原对象。如果\pyth{append}操作失误,相当于没有备份原列表,没办法撤销误操作。如果原列表保存了很重要的数据,使用本方法一定要慎之又慎。
\begin{python}
lst = [1, 2, 3]
lst.append(4)
lst
\end{python}
\subsection{extend}
\pyth{extend()}允许一次增加若干新元素。调用该语法后,原列表对象扩展为新列表对象,也就就地改变。
\begin{python}
a = [1, 2, 3]
b = [4, 5, 6]
a.extend(b)  # a has changed in place

# compare
a + b
a  # a does not change
a = a + b # a has changed

# slicing could be used to do what extend does.
a = [1, 2, 3]
b = [4, 5, 6]
a[len(a):] = b
\end{python}
\subsection{pop}
\pyth{pop()}从列表末尾或指定位置(默认是末尾)删除一个元素,并返回这个元素的值。pop方法是列表方法中唯一一个既修改列表又返回值(非None值)的方法。pop方法可用于实现栈。
\begin{python}
x = [1, 2, 3]
x.pop()
x
x.pop(0)  # remove specific value
x
\end{python}
\subsection{remove}
\pyth{remove()}删除某元素,当列表中有多个元素值相同时,只删除第一次出现的元素。\pyth{remove()}是所谓“就地改变而不返回值(nonreturning in-placechanging)”的方法,它与pop正相反,它修改了列表,但不返回值。这种静默修改,不会给出提示,很容易让人忽略列表发生的改变。因此,要小心使用该方法时,留意它的这种副作用。
\begin{python}
x = ['to', 'be', 'or', 'not', 'to', 'be']
x.remove('be')
x
# note the secone 'be'
\end{python}
\subsection{reverse}
\pyth{reverse()}倒转列表,将列表元素逆序排列。
\begin{python}
x = [1, 2, 3]
x.reverse()
x
\end{python}
\subsection{sort}
\pyth{sort()}方法是就地排序,排序后改变了原列表,没有生成新列表对象。与函数sorted对比,sorted是生成新列表对象。
\begin{python}
  # 1.
  x = [4, 6, 2, 1, 7, 9]
  y = x.sort()
  x
  y # None
  # s is sorted and y is None

  # 2. to reserve the original x, we need to make a copy of x
  x = [4, 6, 2, 1, 7, 9]
  y = x[:]  # make a copy of the original
  y.sort()  # sort
  x
  y

  # 3. y = x is not the option, which binds x,y to the same object
  y=x
  y.sort()
  x
  y

  # 4. Another way is to use `sorted` function
  x
  y = sorted(x)
  x
  y
\end{python}
\subsubsection{扩展排序}
\pyth{sort()}方法有两个可选参数: key, reverse。这两个是关键字参数,设置这两个参数可以改变sort()方法的行为。
\begin{python}
  # 1. 
x = ['aardvark', 'abalone', 'acme', 'add', 'aerate']
x.sort()  
print(x)

# 2.
x = ['aardvark', 'abalone', 'acme', 'add', 'aerate']
x.sort(key=len) # sort by the length of the elements ascent
                # high-order function
print(x)
                 
   # 3.
x = [4, 6, 2, 1, 7, 9]
x.sort(reverse=True)  # descent
print(x)

  # 4.
x = ['aardvark', 'abalone', 'acme', 'add', 'aerate']
x.sort(key=len, reverse=True)  # both by length and descent
print(x)
\end{python}
\section{列表推导}
\begin{itemize}
\item 列表是Python最常用的数据类型,而在列表技术中,列表推导(list comprehension)是常用的编程技术
\item 列表推导就一个目的:生成列表
\item 其技术功能类似循环,却十分简洁,在很多地方取代了循环
\end{itemize}
\subsection{练习:列表推导}
\inputpython{examples/lstcompre.py}{1}{31}
示例6的算法尽管紧凑,但可读性很差。这种情况下,首选用循环来描述算法,因为清晰简洁,可读性、可维护性更好。因此,面对具体问题,选择循环还是列表推导完成任务,不能一概而论,须视情况而定。

\section{字典}
\begin{itemize}
\item 字典是映射数据类型
\item 映射中数据元素值没有整数索引,使用键作为索引访问元素值。
\item “键(key):值(value)”组成的对(pair)称为项(items)。字典是由无序的项组成的集合,要求键必须唯一,不重复,值不必唯一。键只能由不可修改的数据类型充任,比如字符串、数字、元组
\item 字典界定符为花括号{},键值用冒号(:)分隔,项用逗号(,)分隔
\item 如果元组的元素是可改变的数据类型,比如,元组的元素是列表,那么这种元组不能充任键
\item 列表不能充任键。因为,以可改变类型充任键,意味着键值可以修改,一旦修改键,无法保证修改键与字典中的其他键不冲突,有可能破坏键的唯一性。
\end{itemize}
\subsection{创建字典}
使用\pyth{dict()}函数可以将其他字典或(键:值)对序列元素创建为字典。
\begin{python}
  # 1. use {} : , to create a dict
  d = {'Adam' : 24, 'Bob': 19}

  # 2. ues dict() fucntion
  items = [('name', 'Adam'), ('age', 42)]
  d = dict(items)

  # 3. use optional arguments
  # d = dict(name='Adam', age=42) 
  d
  d['name']

  # 4. use list
  k = ['a', 'b', 'c', 'd']
  v = [1, 2, 3, 4]
  di = dict(zip(k,v))
  di

  # 5. create empty dict
  x = dict()
  x = {}
\end{python}
\subsection{查找元素}
字典查找元素是通过查找键,进而查到值。我们熟悉的变量,即是一种字典,通过变量名字访问变量的绑定值。
\begin{python}
a = (1, 2, 3, 4, 5)  #  global
b = 'Hello, world.'  #  global

def demo():
    a = 3            #  local
    b = [1, 2, 3]    #  local
    print('locals:', locals())
    print('globals:', globals())

demo()
\end{python}
通过例子,比较下列表和字典异同。 例:有一本电话簿如下,以列表方式实现查找电话号。
\begin{python}
  names = ['Alice', 'Beth', 'Cecil', 'Dee-Dee', 'Earl','Frank','Gary',
  'Henrry', 'Ivan', 'Jack', 'Karl', 'Leon', 'Monty', 'Nash', 'Owen', 'Peter']
  numbers = ['2341', '9102', '3158', '0142', '5551', '1908', '3392',
  '4872', '1874', '2354', '4081' ,'3158','3201','8282', '6372', 2019']

# find Cecil's number
numbers[2]  # numbers[names.index('Cecil')]

# What's Leon's number?
\end{python}
用列表实现的电话簿,查找电话号方式不符合习惯,人们熟悉的查找方式是用键(名字)找值(电话号),形如:\pyth{phonebook['Cecil']},也希望用这种方式查电话号。使用字典可以达到这种效果。
我们修改例子,以字典方式实现查找电话号。
\begin{python}
  # create a dictionary
  phonebook = {'Alice': '2341', 'Beth':'9102',
    'Cecil': '3258', 'Dee-Dee': '0142', 'Earl': '5551'}

  # find Cecil's and Leon's number
  phonebook['Cecil']
  phonebook['Leon']
\end{python}
\subsection{常用字典函数}
\begin{itemize}
\item \pyth{len(dict)} 返回项个数
\item \pyth{dict[key]} 返回键对应的值
\item \pyth{dict[key] = value} 将值value与键key对应起来,相当于向字典增加一个项
\item \pyth{del dict[key]} 根据键key删除项
\item \pyth{key in dict} 检查以key为键的项是否归属字典dict
\end{itemize}
通过例子和各自的函数,可归纳出字典与列表的区别:
\begin{itemize}
\item 字典的键不必是整数,可以是任意不可修改类型的数据
\item 字典自动增长,依据键给项赋值时,只要字典中尚没有该键,Python会自动给字典创建一项。列表,如果超出索引范围,无法赋值
\item 键必须唯一,值不必唯一
\item 字典归属操作,查找的是键,而非值。列表归属操作value in list查找的是值,而不是索引
\end{itemize}
\subsection{练习:电话簿}
\inputpython{examples/dictfunc.py}{1}{38}
\subsection{字典推导}
字典也有推导式,注意与列表推导的区别。使用字典推导的目的,是为了创建字典。
\inputpython{examples/dictcompre.py}{1}{5}
\subsection{字典方法}
回想下,函数与方法的异同。
\subsubsection{clear}
清除所有字典项。这是个就地(in-place)操作,返回值为None。如果目的是删除字典的所有项,必须使用\pyth{clear()}方法。
\inputpython{examples/dictclear.py}{1}{24}
\subsubsection{copy}
浅复制(shallow copy),复制后,原件、复件变量名称不同,却绑定到同一个值。
\inputpython{examples/dictshallow.py}{1}{8}
通过例子,可以知道:
\begin{itemize}
\item 替换复件中的值,原件不受影响,如username
\item 修改值(就地修改,而非替换),则原件受影响,如machines。因为他们的指向同一个值存储在同一个位置
\item 避免出现浅复制问题的办法是进行深复制(deep copy),拷贝值,同时拷贝值里面嵌套包含的值
\item copy模块中的deepcopy函数完成深复制任务
\end{itemize}
\inputpython{examples/dictdeep.py}{1}{8}
\paragraph{阅读材料}
\begin{itemize}
\item \href{https://docs.python.org/2/library/copy.html}{辨析深浅复制文章1}
\item \href{https://www.quora.com/What-is-deep-copy-and-shallow-copy-in-Python}{辨析深浅复制文章 2}
\end{itemize}
\subsubsection{fromkeys}
\pyth{fromkeys()}根据给定的键,创建新字典,这些键对应的值默认均为None。
\begin{python}
{}.fromkeys(['name', 'age'])

dict.fromkeys(['name', 'age'])

# provide default values
dict.fromkeys(['name', 'age'], '(unknown)')
\end{python}
\subsubsection{get}
\pyth{get()}是访问字典项的方法,它的特点是忽视错误。一般,如果访问的项不在字典里,会报错。而get方法不报错,当然,想报错的话,允许自定义错误提示。
\inputpython{examples/dictget.py}{1}{7}
\subsubsection{练习:用get方法重写电话簿}
\inputpython{examples/dictfunc2.py}{1}{37}
\subsubsection{has\_key}
\begin{itemize}
\item \pyth{has_key()}方法检查字典中是否有某个键。 \pyth{dict.has_key(key)}等价于\pyth{key in dict}
\item 面对多个方法、函数、编程技术可用时,选用适用即可
\item 掌握多种实现方式可以更好地读懂别人的代码
\item 当其他程序员使用按他们兴趣选出的方法、函数,我们能通过读懂代码,理解他们的意图
\end{itemize}
\begin{python}
d = {}
d.has_key('name')
d['name'] = 'Eric'
d.has_key('name')
\end{python}
\subsubsection{items and iteritems}
\pyth{items()}返回一个列表,该列表包含字典所有项,返回值没有特定顺序。\pyth{iteritems()}方法功能基本相同,只不过返回值不是列表,而是循环对象(iterator)。
\begin{python}
  d = { 'title':'Python Web Site', 'usl': 'http://www.python.org', 'spam': 0 }
  d.items()
  it = d.iteritem()
  it
  list(it)  # cast itertor into list
\end{python}
\subsubsection{keys and iterkeys}
\pyth{keys()}返回一个列表,列表内包含字典的键。\pyth{iterkeys()}返回包含字典键的iterator。
\subsubsection{pop}
\pyth{pop()}根据键返回对应的值,同时从字典中删除该项。
\begin{python}
d = {'x':1, 'y':2}
d.pop('x')
d
\end{python}
\subsubsection{popitem}
\pyth{popitem()}从字典中弹出一个值,与列表的\pyth{pop()}方法不同,\pyth{popitem()}弹出的值,没有特定顺序,随意弹出字典一个项。如果想逐一弹出并处理字典项, 使用\pyth{popitem()}正合适。
\begin{python}
d = {
    'title': 'Python Web Site',
    'usl': 'http://www.python.org',
    'spam': 0
}
d.popitem()
d
\end{python}
\subsubsection{setdefault}
\pyth{setdefault()}从字典中据键取值,当键在字典中不存在时,\pyth{setdefault()}可以给该键一个默认值,相当于错误提示。
\begin{python}
d = {}
d.setdefault('name', 'N/A')
d
d['name'] = 'Gumby'
d.setdefault('name', 'N/A')
d
\end{python}
\subsubsection{update}
\pyth{update()}根据键更新值。
\begin{python}
d = {
    'title': 'Python Web Site',
    'usl': 'http://www.python.org',
    'changed': 'Jan 30 11:15:15 Met 2018'
  }
  x = {'title': 'Python Language Website'}
  d.update(x)
  d
\end{python}
\subsubsection{values and itervalues}
\pyth{values()}返回一个列表, 列表包含字典的值。\pyth{itervalues()}返回包含值的iterator。
\begin{python}
d = {}
d[1] = 1
d[2] = 2
d[3] = 3
d[4] = 1
d.values()
\end{python}

\section{元组}
\begin{itemize}
\item 元组和列表一样是序列数据类型,区别在于,元组的元素不可改变)
\item 元组以()为界定符,以逗号(,)为分隔符
\item 创建单元素元组,必须在这单个元素后加“,”,创建元组的符号中,逗号是关键,仅有括号不起作用
\item 序列解包实际上是对元组元素赋值(x, y, z) = 1, 2, 3
\item 元组没有类似列表的方法,因它无须增加、修改、删除。
\end{itemize}
\subsection{创建元组}
\begin{python}
at = ('a',)
atu = ('a', 'b', 'mpilgrim', 'z', 'example')
autp = ()  #  empty tuple

a = (3)
a = 3,  # create a tuple with one single element, comma must be suffixed!

a = 1, 2

# note the difference
3 * (40+2)
3 * (40+2,)

print(tuple('abcdefg'))
al = [-1 ,-4, 6, 7.5, -2.3, 9, -11]
tuple(al)
s = tuple

del s  #  delete tuple, you cannot delete elements of a tuple
\end{python}
\subsection{元组操作}
\begin{itemize}
\item 用索引和切片访问元组元素
\item 元组用途:可用作映射的键;可用作内置函数和方法的返回值
\item 元组中数据一旦定义就不允许更改。访问和处理元组速度比列表快, 如果定义一系列常量值,主要用于遍历元组元素,而不需要对元素进行修改,那么建议使用元组而不用列表
\end{itemize}
\subsubsection{特殊情况}
虽元组元素不可改变,但若元组元素为可变序列,如列表,情况有所不同。
\begin{python}
x = ([1, 2], 3)
x[0][0] = 5
x

x[0].append(8)
x
\end{python}
\subsection{序列解包}
表达式和语句一节,讲过序列解包,再看几个例子。可以对列表、元组等序列类型进行解包。
\begin{python}
keys = ['a', 'b', 'c', 'd']
values = [1, 2, 3, 4]
for k, v in zip(keys, values):
    print(k, v)

vt = (Fasle, 3.5, 'exp')
(x, y, z) = vt
\end{python}
\subsection{生成器推导}
将列表推导、字典推导的语法套用在元组上,写出的推导式,不叫元组推导,而叫作生成器推导。
\begin{itemize}
\item 生成器推导(generator comprehension)使用圆括号界定推导式
\item 列表推导式的结果是列表对象,字典推导的结果是字典,生成器推导式的结果却不是列表,不是字典,也不是元组,而是生成器对象(generator)
\item 生成器是较新的技术
\end{itemize}
\begin{python}
g = ((i+2)**2 for i in range(10))
g
tuple(g)
list(g)

g = ((i+2)**2 for i in range(10))
g.__next__()
g.__next__()
for i in g:
    print(i, end='')
\end{python}

\section{集合}
集合是一种不同于序列和映射的数据类型。集合中元素是无序的,元素不重复,唯一。集合使用花括号{}做界定符。
\subsection{创建、修改、删除}
\begin{python}
  a = {3, 5}  # create use curly braces
  aset = set(range(8, 14))  # use set()

  bset = set([0, 1, 2, 3, 0, 1, 2, 3, 7, 8]) # delete the repeats while create setd 
  c = set()  #  empty set
  
  a = {1, 4, 2, 3}
  a.pop()  
  a.add(7) 
  a.remove(3)  
  a.clear() 
  
  del a 
\end{python}
\subsection{集合运算}
集合支持交、并、差集等运算。
\begin{python}
a = set([8, 9, 10, 11, 12, 13])
b = set([0, 1, 2, 3, 7, 8])

a | b   
a.union(b)

a & b    
a.intersection(b)  
a.difference(b) 
a.symmetric_difference(b) 

x = {1, 2, 3}
y = {1, 2, 5}
z = {1, 2, 3, 4}

#  comparision
x < y
x < z
y < z

#  is sub set
x.issubset(y)
x.issubset(z)
\end{python}
\subsection{练习:生成随机数}
要求:取100个介于0-9999之间的随机数。
\begin{python}
from random import randint
listRandom = [randint(0, 9999) for i range(100)]

noRepeat = []
for i in listRandom:
    if i not in noRepeat:
        noRepeat.append(i)

len(listRandom)
len(noRepeat)

# By using set, one line of code do the job
newSet = set(listRandom)
\end{python}

\section{再学字符串}
字符串是不可修改序列数据类型(Strings are immutable sequence of Unicode points)。
\subsection{基本操作}
所有序列操作索引、切片、倍增、归属、求长度、最大、最小值等均适用于字符串。但所有赋值操作均不适用,因为字符串不可修改。
\begin{python}
website = 'http://www.python.org'
website[-3:] = 'com'  # Type error
\end{python}
\subsection{格式化字符串}
\subsubsection{传统方式}
字符串格式化适用格式化操作符, 百分号(\%)。
\begin{itemize}
\item 百分号左侧放待格式化字符串,右侧放置待格式化的值
\item 常见格式字符,有\%s,\%d, \%e, \%f, \%\%等。
\item 值可以是单个值,如字符串、数字,也可以是元组和字典,元组常用
\end{itemize}
\begin{python}
format = "Hello, %s. %s enough for ya?"
values = ('world', 'Hot')
print(format % values)

s = "Pi with three decimals: %.3f"
from math import pi
print(s % pi)
\end{python}
上例中,\%s部分称为转换部分(conversion specifiers)。完整的转换部分包括5个成分,其值和顺序决定了以何种格式输出。
\begin{itemize}
\item \%标识转换部分开始
\item 转换标志flag,决定是否左对齐,数值是否带正负号
\item 最小宽度
\item 精度
\item 转换类型
\end{itemize}
\subsubsection{练习:格式化输出字符串}
\inputpython{examples/strfmtpct.py}{1}{46}
\subsubsection{主流方式}
\begin{itemize}
\item 使用\pyth{str.format(argument list)}方法格式化字符串
\item 该方法使用{}和:代替传统的\%对字符串进行格式化
\item 该方法可以使用位置、参数名字对字符串进行格式化,且支持序列解包
\end{itemize}
\inputpython{examples/strfmt.py}{1}{65}
\subsubsection{新出现方式f-string}
这部分内容留给同学们自学。
\href{https://docs.python.org/3/reference/lexical_analysis.html#f-strings}{f-string}
\subsubsection{字典的格式化}
这里顺道讲下字典这种数据类型的格式化输出。
\begin{python}
- phonebook = {'Alice': '2341', 'Beth':'9102', 'Cecil': '3258',
  'Dee-Dee': '0142', 'Earl': '5551'}
# find the number of Cecil's
- print("Cecil's phone number is %(Cecil)s." % phonebook)
\end{python}
\subsection{字符串方法}
字符串的方法比列表方法更丰富,语法为\pyth{object.method(arguments)}。
\subsubsection{find}
在字符串中查找子串,若找到返回子串的第一个字符索引,若没找到则返回-1。
\inputpython{examples/strfind.py}{1}{6}
本节给出一些与所讲字符串方法功能相近,又有一些细微差别的方法。留给同学们自学。
\pyth{rfind(), index(), rindex(), count(), startswith(), endswith()}
\subsubsection{replace}
替换字符串。
\pyth{'This is a test'.replace('is', 'eez')} \\
自学: \pyth{translate(), expandtabs()}
\subsubsection{join}
非常重要的方法,与\pyth{split()}互为反操作,将序列元素拼接为字符串。
\inputpython{examples/strjoin.py}{1}{13}
\subsubsection{split}
非常重要方法,与\pyth{join()}互为反操作,将字符串分割为序列。
\inputpython{examples/strsplit.py}{1}{3}
自学: \pyth{rsplit(), splitlines()}
\subsubsection{lower}
返回字符串小写形式。
\inputpython{examples/strlower.py}{1}{9}
自学:\pyth{islower(), capitalize(), swapcase(), title(), istitle(), upper(), isupper()}
\subsubsection{strip}
清除字符串左端和右端空白,或指定字符。仅清除两端,字符串中间的空白或字符不受影响。
\begin{python}
'internal whitespace is kept '.strip()
'*** SPAM * for * every one!!!***'.strip('*!') #  delete *!
\end{python}
自学:
\pyth{lstrip(), rstrip()}
\section{其他高级数据结构}
Python支持许多高级数据结构,只是这些数据结构不再是内置的,而由工具包支持。使用这些数据结构,需要引入相应的包。较常使用的有如下几种。
\begin{itemize}
\item 堆 \pyth{import heapq}
\item 队列 \pyth{import Queue}
\item 栈和链表,用列表模拟
\item 其他高级数据类型,通过\pyth{import collectoins}引入\href{https://docs.python.org/3/library/collections.html?highlight=collections#module-collections}{容器类型}
\end{itemize}
